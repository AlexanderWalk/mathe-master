\documentclass{article}
\usepackage{mathtools}
\usepackage{amsfonts}
\usepackage{enumerate}
\usepackage[utf8]{inputenc}
\usepackage[ngerman]{babel}

\newtheorem{theorem}{Satz}[section]
\newtheorem{subtheorem}{Satz}[theorem]
\newtheorem{example}{Beispiel}[theorem]


\title{Mathematik (Master) für Informatik Skript}
\date{SoSe 2021}
% Bei PR gerne Namen hinzufügen, wenn du willst
\author{erstellt von Prof. Dr. Preisenberger\\ transkribiert von Niklas Stich, Alexander Walk}

\begin{document}
\begin{titlepage}
\maketitle
\tableofcontents
\end{titlepage}


\section{Komplexe Zahlen und komplexwertige Funktionen}
\subsection{Historisches}
16. Jahrhundert: Lösung algebraischer Gleichungen\\
Gegeben: \(a_{k}\in\mathbb{Q},\quad k=0,\dots,n\in\mathbb{N}_{0}\)
\begin{theorem}
\begin{align}
    a_{n}x^{n}+a_{n-1}x^{n-1}+\dots+a_{1}x+a_{0} &= 0, \quad a_{n}\neq 0 \\
    \sum_{k=0}^{n}a_{k}^{k} &= 0 \\
    \underbrace{p(x)}_\text{Polynom n-ten Grades}&=0, \quad p \in \mathbb{Q}[x]
\end{align}
\end{theorem}
Gesucht: Lösung von (1.1) zunächst für $n \leq 3$\\
bekannt war:
\begin{enumerate}[(i)]
    \item $n=1,\quad a_{1}x+a_{0}=0,\quad x=-\dfrac{a_{0}}{a_{1}}$
    \item $\begin{aligned}[t]
        n&=2: & a_{2}&x^{2}+a_{1}x+a_{0} & &=0\\
        a_{2}&\neq 0, & &x^{2}+\dfrac{a_{1}}{a_{2}}x+\dfrac{a_{0}}{a_{2}} & &= 0\\
        & & &x^{2}+2\dfrac{a_{1}}{2a_{2}}x & &= - \dfrac{a_{0}}{a_{2}}\\
        & & &x^{2}+2\dfrac{a_{1}}{2a_{2}}x +(\dfrac{a_{1}}{2a_{2}})^2 & &= (\dfrac{a_{1}}{2a_{2}})^2 - \dfrac{a_{0}}{a_{2}}\\
        & & (&x+\dfrac{a_{1}}{2a_{2}})^2 & &= (\dfrac{a_{1}}{2a_{2}})^2 - \dfrac{a_{0}}{a_{2}}\\
        & & &x+\dfrac{a_{1}}{2a_{2}} & &=\pm\sqrt{\dfrac{a_{1}^{2}}{4a_{2}^{2}} - \dfrac{4a_{2}a_{0}}{4a_{2}^{2}}}\\
        & & &x_{1/2} & &=-\dfrac{a_{1}}{2a_{2}}\pm\dfrac{\sqrt{a_{1}^{2}-4a_{0}a_{2}}}{2a_{2}}
        \end{aligned}$\\
		Falls $\begin{aligned}[t]
            &\Delta = a_{1}^{2}-4a_{0}a_{2} & &< 0 & &\Rightarrow \text{keine reelle Lösung}\\
            &\Delta & &=0 & &\Rightarrow \text{genau eine Lösung}\\
            &\Delta & &> 0 & &\Rightarrow \text{genau zwei Lösungen}
        \end{aligned}$\\
        Die Lösungen sind die Schnittpunkte einer Parabel mit der x-Achse. Die Schnittpunkte existieren für Parabeln mit $\Delta \geq 0$.
        Aber für ${a_{2}=1,\,a_{0},\,a_{0}=-2}$ ergibt sich: $x_{1/2} = \pm \sqrt{2}\notin \mathbb{Q}$.\\
        Folge: Erweiterung von $\mathbb{Q}$ zu $\mathbb{Q}(\sqrt{2})=\{a + b\sqrt{2}\,|\,a,b\in\mathbb{Q}\}$.
        $\mathbb{Q}(\sqrt{2})$ ist ein Körper, in dem wie in $\mathbb{Q}$ gerechnet werden kann und in dem die Lösungen von $x^{2}=2$ existieren.
    \item Für Gleichungen 3. Grades keine allgemeine Lösungsformel, jedoch für spezielle Gleichungen\\\\
        Cardano: \\
        Für bestimmte Gleichungen 3. Grades können alle ihre reellen Lösungen mit Lösungen von $x^{3}=\alpha$ und $x^{2}=\beta=-1$ beschrieben werden,
        sofern man mit diesen Lösungen wie in $\mathbb{R}$ rechnet.\\\\
        Ergebnis: Einführung einer neuen Zahl i mit $i^{2} = -1$
        Damit waren die imaginären bzw. komplexen Zahlen erfunden.\\
        $\rightarrow$ Blüte des 19. Jahrhunderts 
    \item Galois, Abel (19. Jht.)\\
        Für die Lösungen der Gleichung $\sum_{k=0}^{n}a_{k}x^{k} = 0$ gibt es nur für $n\leq4$ allgemeingültige,
        von den Koeffizienten $a_{k}$ abhängige, Lösungsformeln.\\
        Für $n\geq5$ gibt es \underline{keine} allgemeingültigen Formeln.
\end{enumerate}
\subsection{Der Körper der komplexen Zahlen \quad \; (Hartmann 5.3)}
    Wir definieren: $\mathbb{C} = \{ (a,b) | a,b \in \mathbb{Q}\}$\\
    Gleichheit: $(a,b),(c,d) \in \mathbb{C}, (a,b)=(c,d) \Leftrightarrow a=c, b=d\\
    "\!+\!":\mathbb{C}\times\mathbb{C}\rightarrow\mathbb{C},((a,b),(c,d))\mapsto ((a,b)+(c,d))=(a+c,b+d)\\
    "\ast":\mathbb{C}\times\mathbb{C}\rightarrow\mathbb{C},((a,b),(c,d))\mapsto ((a,b)\ast(c,d))=(a\ast c-b\ast d,a\ast d+b\ast c) $
\setcounter{theorem}{2}
\begin{subtheorem}\,\\
    Es gilt $(\mathbb{C},"+","\ast")$ ist ein Körper, d.h.
    \begin{enumerate}
        \item $(\mathbb{C},"+")$ ist eine abelsche Gruppe mit dem neutralen Element $(0,0) = 0_{\mathbb{C}}$
        \item $(\mathbb{C}\setminus \{0_{\mathbb{C}}\},"\ast")$ ist eine abelsche Gruppe mit dem neutralen Element\\
         $(1,0)=1_{\mathbb{C}}$;\, inverses Element: weiter unten.
         \item Distributivgesetz: $\forall x,y,z\in\mathbb{C}:(x+y)\ast z=x\ast z+y\ast z$\\
         Beweis: Nachrechnen als Übung.
    \end{enumerate}
\end{subtheorem}
Wir sehen sofort: $M=\{(x,0)|x\in \mathbb{R}\}\subset \mathbb{C}$ bildet mit $"+"$ und $"\ast"$ in $\mathbb{C}$ einen Unterkörper
und $\Phi :M\rightarrow\mathbb{R},(x,0)\mapsto \Phi((x,0))=x$ ist eine Bijektion.\\
Außerdem gilt:\\
$\Phi ((a,0)+(b,0))=\Phi ((a,0))+\Phi ((b,0))\\
\Phi((a,0)\ast (b,0)) = \Phi((a,0))\ast \Phi((b,0))\\
\Phi$ ist also ein Körperisomorphismus (Hartmann) $M\widehat{=}\mathbb{R}$.
Wir finden also mit $M$ die reellen Zahlen in $\mathbb{C}$ wieder.\\
Insbesondere entspricht der Gleichung $x^{2}=-1$ in $\mathbb{C}$ die Gleichung\\
$(a,b)\ast (a,b)=(-1,0)$\\
Wie erwartet, gibt es für diese Gleichung in $\mathbb{C}$ eine Lösung\\
\\
$\underline{\text{Lösung der Gleichung}\;(a,b)\ast (a,b)=(-1,0)}$\\
Ansatz: $(a,b)\ast (a,b) = (a^{2}-b^{2},2ab)=(-1,0)$\\
Komponentenvergleich: $a^{2}-b^{2}=-1,\,2ab=0$\\
Man erhält: $a=0, b=\pm 1\Leftrightarrow (0,\pm 1)\in\mathbb{C}$\\
Wir setzen $i=(0,1),\; -i=(0,-1)$\\
$i$ und $-i$ sind die Lösungen der Gleichung in $\mathbb{C}$\\
\\
$\underline{\text{Realteil und Imaginärteil einer kompelxen Zahl}}$\\
\\
Geg: $z=(x,y)\in \mathbb{C}\quad (w=(u,v)\in \mathbb{C})$\\
Wir nennen $x:\text{Realteil von z;}\; Re(z)=x\in\mathbb{R}$\\
Wir nennen $y:\text{Imaginärteil von z;}\; Im(z)=y\in\mathbb{R}$\\
\\
\setcounter{example}{1}
\begin{example}\,\\
    $Re(1,0)=1,\quad Im(1,0)=0\\
    Re(1,3)=1,\quad Im(1,3)=3\\
    w=(u,v),\quad Re(w)=u,\, Im(w=v)$
\end{example}
\underline{Vereinfachung der Notation}\\
Wir betrachten\quad $z=(x,y)\in\mathbb{C},\quad (x,0),(y,0)\in M\cong\mathbb{R},\;(0,1)=i$\\
und rechnen leicht nach:\\
$z=(x,0)+\underbrace{(0.1)\ast (y,0)}_{\underbrace{(0\ast y-1\ast 0,0+y)}_{(0,y)}}=(x,y)$\\
Wir kürzen ab: $x\widehat{=}(x,0), y\widehat{=}(y,0),i\widehat{=}(0,1)$\\
Wir erhalten damit \underline{z=x+i*y}\\
Bemerkung: $\underbrace{1\widehat{=}(1,0)}_{=1_{\mathbb{C}}}:\quad z=x\ast 1_{\mathbb{C}}+y\ast i,\quad x,y\in\mathbb{R}$\\
Im Weiteren verwenden wir für die Darstellung einer komplexen Zahl \\
$z=(x,y)\in\mathbb{C}$:\quad$z=x+iy\;\text{mit}\; x,y\in\mathbb{R},\; i=(0,1),\; i^{2}=(-1,0)$\\
Wir beschreiben diese Darstellung der komplexen Zahlen als \underline{karthesische Darstellung}\\
\\
\underline{Rechnen mit komplexen Zahlen in kartesischer Darstellung}
Seien $w=u+iv,\; z=x+iy\;$zwei komplexe Zahlen.
\begin{enumerate}
    \item Gleichheit komplexer Zahlen\\
          w und z sind gleich, wenn gilt:\quad$\underline{u=x,\; v=y}$\\
    \item Addition und Multiplikation\\
          Für Addition $"+"$ und Multiplikation $"\ast"$ von w und z gilt:\\
          $\underline{w+z=(u+x)+i(v+y)}\\
          \underline{w\ast z=(ux - vy)+i(uy+vx)}$
    \item To be continued
\end{enumerate}
\end{document}
